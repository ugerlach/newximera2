\documentclass{ximera}

\usepackage{color,hyperref}
\begin{document}
\begin{center}
                           \textbf{ COURSE ANNOUNCEMENT}\\
                             Mathematics 5756 (Autumn)\\
			     Mathematics 5756 (Spring)\\
\end{center}
\begin{enumerate}
\item[Course Name:]
Modern Mathematical Methods in Relativity Theory I, II
               (a.k.a. ``Applied Differential Geometry'')
\item[Class Number:] 
  32188 or 32197
\item[Time:] 
  Autumn 2014: MWF 12:40pm
\item[Credits:]
  3 per semester
\item[Prerequisites:]
  \begin{enumerate}
    \item[] 
      Multivariable differential calculus and linear algebra
      (e.g. Math 2568 and/or 5101).
    \item[]
      A physics course (e.g. Physics 133 or higher).
    \item[]
	No prior knowledge of tensor calculus is assumed.
               However, we do assume a mature attitude towards
               mathematics and physics.
  \end{enumerate}
\item[Audience:] 
  Undergraduate and graduate
\item[Goal and Purpose:] Methodically develop the fundamental
  mathematical concepts and methods responsible for the successes in
  20$^{th}$ century physics, mathematics, and theoretical engineering.
  Thus Math 5756 concretizes these developments in terms of 
  \begin{enumerate}
    \item
      Special Relativity as the cognitive bridge to 20$^{th}$ century geometry,
    \item
      multilinear algebra as a source of geometrical structures,
    \item
      linear algebra's marriage to multi-variable calculus
    \item
      differential geometry as a three level hierarchy characterized by its
    \begin{itemize}
      \item
        Differential structure
      \item
        Parallel transport structure (a.k.a. covariant derivative)
      \item
        Metric structure
    \end{itemize} 
    \item
      the exterior calculus
    \item
      Cartan's two structural equations of differental geometry and their application to
    \item
      the Cartan-Misner calculus.
  \end{enumerate}
\item[Agenda:]

  \begin{enumerate}
     \item[a)]   Assimilate the mathematical chapters
                   of our primary text (``\textit{Gravitation}'' by MTW,
                   see references below), thus
                   to develop an appreciation and the modern machinery for
                   the mathematical framework of the laws of physics from
                   the spacetime perspective.
                   The development will focus on
                   \begin{enumerate}
                     \item[(1)]
                       the underlying differential geometric
                       framework of spacetime, and
                     \item[(2)]
                       the formulation (arising from classical mechanics,
                   fluid dynamics, and wave mechanics) of its properties.
                   \end{enumerate}

    \item[b)]  Show why and how mathematics
                   is the language of physical science, in particular
                   of those aspects of physics dealing with processes of
                   extreme violence (relativistic hydrodynamics,
                   relativistic laser-matter interaction,
                   high energy density physics,
                   gravitational collapse in flat or
                   curved spacetimes).
  \end{enumerate}
\item[Website:]
\url{http://www.math.ohio-state.edu/~gerlach/math5756}
\end{enumerate}
\begin{center}
                                    DESCRIPTION
\end{center}
\begin{enumerate}
\item[Math 5756 (Autumn):]
  \begin{itemize}
    \item  A rapid course in special relativity: spacetime geometry, 
      event horizons and accelerated frames;
    \item  tensors, metric geometry vs symplectic geometry;
    \item  exterior calculus, Maxwell field equations;
    \item  manifolds, Lie derivatives, and Hamiltonian dynamics in
           phase space;
    \item  parallel transport, torsion, tensor calculus;
    \item  curvature and Jacobi's equation of geodesic deviation;
    \item   Cartan's two structural equations, metric induced properties,
           and Cartan-Misner curvature calculus.
  \end{itemize}
\item[Math 5757 (Spring):]
  \begin{itemize}
    \item Geodesics: Hamilton-Jacobi theory, the principle of constructive
      interference;
    \item  stress-energy tensor: hydrodynamics in curved
           spacetime and Einstein field equations; 
    \item
      The conservation laws and the Bianch identities mathematized in terms of the ``Boundary of a Boundary is zero ($\partial\,\partial\, \Omega =0$)'' Principle.
    \item  Solutions to the Einstein's field equations: stars, black holes, gravitational collapse,  geometry and  dynamics of the universe; 
    \item vector harmonics, tensor harmonics, acoustic and
    gravitational waves in violent relativistic backgrounds.
  \end{itemize}
\item[Textbooks:]
  \begin{enumerate}
    \item
      \emph{Gravitation}  by C. W. Misner, K. S. Thorne, and J. A. Wheeler.
    \item
      Selections from \emph{Mathematical Methods of Classical Mechanics}
                by V. I. Arnold.
    \item
      Selections from \emph{Lecture Notes on Elementary Topology and
                Geometry}  by I. M. Singer.
    \item
      Selections from \emph{Spacetime Physics}, 2nd edition, by E. Taylor
                and J.A. Wheeler
  \end{enumerate}
\end{enumerate}

I am glad to answer any questions.

Ulrich Gerlach (Tel.: 292-7235; Office: MW 506)\\
~\\
\end{document}
\begin{enumerate}
\item[Telephone:] 292-2560 or 292-3572
\item[FAX:] 292-1479
\item[e-mail:]  \href{gerlach@math.ohio-state.edu}{gerlach@math.ohio-state.edu}
\end{enumerate}

\end{document}
